\documentclass{article}
\usepackage[utf8]{inputenc}
\usepackage[noglossbreaks]{covington}
\usepackage{tipa}
\usepackage{amssymb}
\usepackage{booktabs}
\usepackage{geometry}
 \geometry{
 a4paper,
 total={170mm,257mm},
 left=20mm,
 top=20mm,
 }

\newcommand{\h}{{$^h$}} 
\newcommand{\R}{{\*r}}
\newcommand{\N}{{$\varnothing$}}

\title{50 Noun Phrases}
\author{Alison Hau\vspace{-2ex}}
\date{}

\begin{document}
\maketitle
 
\section{Pronouns}
Pronouns have \textbf{number} (singular, dual, plural), \textbf{person} (1st, 2nd inclusive, 2nd exclusive, 3rd), and \textbf{case} (nominative, accusative, and adpositive (used in adpositional phrases)).  Meeple (mer-people) do not have gender or sex, as is reflected in the language, so third person singular pronouns are applied the same to all individuals and also things. The pronouns are shown in Table \ref{table:1}.

\begin{table}[h]
\centering
\begin{tabular}{r l c c c }

\toprule
	Number & Person & Nominative & Accusative & Adpositive \\
\midrule
	SG & 1 & \textipa{wlAt\h{}} & \textipa{wlAnA\R{}} & \textipa{wlAnA} \\ 
	& 2 & \textipa{zEIn} & \textipa{zEInA\R{}} & \textipa{zEInA} \\
	& 3 & \textipa{Za} & \textipa{ZaA\R{}} & \textipa{zaA} \\
\midrule
	DU & 1 & \textipa{slAp\h{}} & \textipa{slAmi\R{}} & \textipa{slAmi} \\
	& 2 inc & \textipa{slAn!} & \textipa{slAni\R{}!} & \textipa{slAni!} \\
	& 2 exc & \textipa{zEa} & \textipa{zEai\R{}} & \textipa{zEai} \\
	& 3 & \textipa{ZEnt\h{}} & \textipa{ZEn:i\R{}} & \textipa{ZEn:i} \\
\midrule
	PL & 1 & \textipa{nES} & \textipa{nES\R{}} & \textipa{nES@} \\
	& 2 inc & \textipa{nES!} & \textipa{neS\R{}!} & \textipa{nES@!} \\
	& 2 exc & \textipa{sEi} & \textipa{sEi\R{}} & \textipa{sEi@} \\
	& 3 & \textipa{ZEint\h{}} & \textipa{ZEin:\R{}} & \textipa{ZEin@} \\
\bottomrule

\end{tabular}
\caption{Table of Pronouns}
\label{table:1}
\end{table}

\section{Nominal Cases}
Nouns have \textbf{number} (singular, dual, plural) and \textbf{case} (nominative, accusative, possessive, and adpositive (used in adpositional phrases)). The number and case are indicated with suffixed morphemes, where each combination of number and case has a different morpheme. The nominative singular form is marked with a null morpheme. 

\begin{table}[h]
\centering
\begin{tabular}{r l c c c }

\toprule
	Number & Nominative & Accusative & Possessive & Adpositive \\
\midrule
	SG & \textipa{-\N{}} & \textipa{-A\R{}} & \textipa{It\h{}} & \textipa{-A} \\ 
\midrule
	DU & \textipa{-m} & \textipa{-i\R{}} & \textipa{-wE} & \textipa{-i} \\
\midrule
	PL & \textipa{-He} & \textipa{-\R{}} & \textipa{-sfE} & \textipa{-@} \\
\bottomrule

\end{tabular}
\caption{Case endings}
\label{table:2}
\end{table}


\subsection{Accusative Case}
The accusative case is most often used without an adposition to indicate a noun is the object of a transitive verb.

\textit{e.g.}
\begin{enumerate}
	\item
		\trigloss[preamble={\textipa{saIA\R{} SlImsA\R{}} }]
	{\textipa{saI} -\textipa{A\R{}} \textipa{SlIms} -\textipa{A\R{}} }
	{tail-fin -\textsc{acc.sg} red -\textsc{acc.sg}}
	{tail-fin {} red {} }
	{red tail-fin}
\end{enumerate}

\subsection{Nominal Possession (Possessive Case)}
Possession is shown by marking the possessive case on the possessed, with no marking on the possessor.  Possessed nouns can be \textbf{alienable} or \textbf{inalienable}, which affects the word order within the noun phrase.  Alienable nouns appear before the possessor, and inalienable nouns appear after the possessor.

\textit{e.g.}

\begin{enumerate}
	\item 
	    \trigloss[preamble={\textipa{saIp\h{}  NEwHI:t\h{}}}]
		{\textipa{saIp\h{}} -\N{} \textipa{NEwHI} \textipa{-It\h{}} }
		{tail-fin -\textsc{sg.nom} color -\textsc{sg.poss} }
		{tail-fin {} its-color {} }
            	{tail-fin's color}
	\item 
		\trigloss[preamble={\textipa{S:eS:It\h{} wlat\h{} }}]
		{\textipa{S:eS:} \textipa{-It\h{}} \textipa{wlat\h{}} }
		{fish -\textsc{sg.poss} \textsc{1.sg.pro} }
		{its-fish {} it }
            	{its fish}
	\item 
		\trigloss[preamble={\textipa{wlAt\h{} HEmWIt\h{}} }]
		{\textipa{wlAt\h{}} \textipa{HEmW} \textipa{-It\h{}} }
		{\textsc{1.sg.pro} head -\textsc{sg.poss} }
		{it its-head {}}
            	{its head}
	\item 
	    \trigloss[preamble={\textipa{saIp\h{}  NEwHI:t\h{}}}]
		{\textipa{saIp\h{}} -\N{} \textipa{NEwHI} \textipa{-It\h{}} }
		{tail-fin -\textsc{sg.nom} color -\textsc{sg.poss} }
		{tail-fin {} its-color {} }
            	{tail-fin's color}
\end{enumerate}

In the examples above, a fish is not inalienable to a merson (mer-person), but color is inalienable to a tail-fin and a merson's head is inalienable to a merson.  Comparison between examples 2 and 3 demonstrates the word order difference between alienable and inalienable possessives.

Nominal possession can be applied recursively, where the possession is applied first to the head, then to each possessor. A brief example:

\begin{enumerate}
	\item 
		\trigloss[preamble={\textipa{S:eS:It\h{} NEwHI:t\h{}} wlAt\h{}}]
		{\textipa{S:eS:} -\textipa{It\h{}} \textipa{NEwHI} -\textipa{It\h{}} \textipa{wlAt\h{}} }
		{fish -\textsc{sg.poss} color -textsc{sg.poss} \textsc{1.sg.pro} }
		{its-fish {} its-color {} me }
            	{my fish's color}
\end{enumerate}

For more examples of recursion, see Section \ref{sec:recursion}.

\subsection{Adpositive Case}
This case is taken by nouns in adpositional phrases. The adposition is in most cases suffixed (postposition) to the last word in the adpositional phrase.  These are can be used used to indicate a location, position, indirect object, or more.

\textit{e.g. indirect object}
\begin{enumerate}
	\item 
		\trigloss[preamble={\textipa{S:eS:Amat\h{}}}]
		{\textipa{S:eS:} -\textipa{A} -\textipa{mat\h{}} }
		{fish -\textsc{sg.adp} -\textsc{io.post} }
		{fish {} to }
            	{to a fish}
	\item
		\trigloss[preamble={\textipa{S:eS:imat\h{}}}]
		{\textipa{S:eS:} -\textipa{i} -\textipa{mat\h{}} }
		{fish -\textsc{du.adp} -\textsc{io.post} }
		{fish two to }
            	{to two fish}
\end{enumerate}

\textit{e.g. instrument}
\begin{enumerate}
	\item 
		\trigloss[preamble={\textipa{anIlziAsi}}]
		{\textipa{anIlzi} -\textipa{A} -\textipa{si} }
		{whale-bone -\textsc{sg.adp} -\textsc{instr.post} }
		{whale-bone {} using }
            	{using a whale-bone}
	\item
		\trigloss[preamble={\textipa{anIlzi:si}}]
		{\textipa{anIlzi} -\textipa{i} -\textipa{si} }
		{whale-bone -\textsc{du.adp} -\textsc{instr.post} }
		{whale-bone two using }
            	{using two whale-bones}

\end{enumerate}

\textit{e.g. in}
\begin{enumerate}
	\item 
		\trigloss[preamble={\textipa{S:eS:As\OE}}]
		{\textipa{S:eS:} -\textipa{A} -\textipa{s\OE} }
		{fish -\textsc{sg.adp} -\textsc{ine.post} }
		{fish {} in }
            	{in a fish}
	\item
		\trigloss[preamble={\textipa{S:eS:@ slIx@s\OE}}]
		{\textipa{S:eS:} -\textipa{@} \textipa{slIx} -\textipa{@} -\textipa{s\OE} }
		{fish -\textsc{pl.adp} some -\textsc{pl.adp} -\textsc{ine.post} }
		{fish {} some {} in }
            	{in some fish}

\end{enumerate}

\textit{e.g. into}
\begin{enumerate}
	\item 
		\trigloss[preamble={\textipa{S:eS:Asei}}]
		{\textipa{S:eS:} -\textipa{A} -\textipa{sei} }
		{fish -\textsc{sg.adp} -\textsc{ill.post} }
		{fish {} in }
            	{into a fish}
	\item
		\trigloss[preamble={\textipa{S:eS:@ slIx@sei}}]
		{\textipa{S:eS:} -\textipa{@} \textipa{slIx} -\textipa{@} -\textipa{sei} }
		{fish -\textsc{pl.adp} some -\textsc{pl.adp} -\textsc{ill.post} }
		{fish {} some {} in }
            	{into some fish}

\end{enumerate}


\section{Adjectives}
Adjectives modifying nouns appear after the nouns they modify and agree in number and case.

\textit{e.g.}
\begin{enumerate}
	\item
\trigloss[preamble={\textipa{S:eS: SlIms} }]
	{\textipa{S:eS:} -\N{} \textipa{SlIms} -\N{} }
	{fish -\textsc{nom.sg} red -\textsc{nom.sg}}
	{fish {} red {} }
	{red fish}
	\item
\trigloss[preamble={\textipa{S:eS:m SlImsm} }]
		{\textipa{S:eS:} -\textipa{m} \textipa{SlIms} -\textipa{m} }
	{fish -\textsc{nom.du} red -\textsc{nom.du}}
	{fish {} red {} }
	{two red fish}

\end{enumerate}

\section{Numbers and Plurals}

\subsection{Plurals}
Adjectives agree with the nouns the modify in number and case, and they take the same case endings as nouns.  For singular and dual nouns, if no number is included then it is an implicit one or two respectively.  For plural nouns, either a number must be specified or the word \textipa{slIx}, meaning \textit{some unspecified amount}, must be used.\\
\\
\textit{e.g. red tail-fin(s)}

\begin{enumerate}
	\item 
		\trigloss[preamble={\textipa{saIp\h{} SlIms} }]
	{\textipa{saIph\h{}} -\N{} \textipa{SlIms} -\N{} }
	{tail-fin -\textsc{nom.sg} red -\textsc{nom.sg}}
	{tail-fin {} red {} }
	{red tail-fin}
	\item 
		\trigloss[preamble={\textipa{saIm: SlImsm} }]
	{\textipa{saIm} -\textipa{m} \textipa{SlIms} -\textipa{m} }
	{tail-fin -\textsc{nom.du} red -\textsc{nom.du}}
	{tail-fin -s red {} }
	{(two) red tail-fins}
	\item
		\trigloss[preamble={\textipa{saImHe SlImsHe saHe} }]
	{\textipa{saIm} -\textipa{He} \textipa{SlIms} -\textipa{He} \textipa{sa} -\textipa{He} }
	{tail-fin -\textsc{nom.pl} red -\textsc{nom.pl} three -\textsc{nom.pl}}
	{tail-fin -s red {} three {}}
	{three red tail-fins}
	\item 
		\trigloss[preamble={\textipa{saImHe SlImsHe slIxHe} }]
	{\textipa{saIm} -\textipa{He} \textipa{SlIms} -\textipa{He} \textipa{slIx} -\textipa{He} }
	{tail-fin -\textsc{nom.pl} red -\textsc{nom.pl} some -\textsc{nom.pl}}
	{tail-fin -s red {} some {}}
	{some red tail-fins}
\end{enumerate}

\subsection{Numbers}
The number system of this language (name undetermined) is in base 12, because each hand has 6 fingers. Numbers 0-12 have their own words, numbers 13-15 are compounds with some irregularity, and all numbers beyond are also compounds formed the same way but regularly.  Compounds take the form \textit{number + } \textipa{He} \textbf{(\textit{and} used for adding numbers) + number}.  24 is its own word, and other multiples of 12 are compounds of the form \textit{k twelve}.

\begin{table}[h]
\centering
\begin{tabular}{l l l }

\hline
    & Language (IPA) & English (literal) \\
\cline{2-3}
1       &  \textipa{HU}   & one      \\
2       & \textipa{ak$^h$}    & two       \\
3       & \textipa{sa}    & three      \\
4       & \textipa{NeZ}     & four      \\
5       & \textipa{mUt$^h$}      & five       \\
6       & \textipa{\OE mp$^h$}   & six      \\
7       & \textipa{ei@}        & seven       \\
8       & \textipa{esl}     & eight      \\
9       & \textipa{Hi}     & nine      \\
10      & \textipa{iR$^h$}      & ten       \\
11      & \textipa{fUf}   & eleven      \\
12       & \textipa{at$^h$}        & twelve       \\
13       & \textipa{anHeU}      & twelve and one      \\
14       & \textipa{anHea}      & twelve and two      \\
15       & \textipa{anHesa}       & twelve and three       \\
16       & \textipa{anHeNeZ}     & twelve and four      \\
17       & \textipa{anHemUt$^h$}        & twelve and five       \\
\hline
\end{tabular}
\caption{Some number examples}
\label{table:3}
\end{table}

Numbers modifying nouns agree with the nouns they modify in number and case and appear after all adjectives modifying that noun.

\textit{e.g. four hours}
\begin{enumerate}
	\item
		\trigloss[preamble={\textipa{xEnHe NeZHe} }]
	{\textipa{xEn} -\textipa{He} \textipa{NeZ} -\textipa{He} }
	{hour -\textsc{nom.pl} four -\textsc{nom.pl}}
	{hours {} four {} }
	{four hours(\textsc{nominative})}
	\item
		\trigloss[preamble={\textipa{xEn\R{} NeZ\R{}}  }]
	{\textipa{xEn} -\textipa{\R{}} \textipa{NeZ} -\textipa{\R{}} }
	{hour -\textsc{acc.pl} four -\textsc{acc.pl}}
	{hours {} four {}}
	{four hours(\textsc{accusative})}
	\item
		\trigloss[preamble={\textipa{xEnsfE NeZsfE} }]
	{\textipa{xEn} -\textipa{sfE} \textipa{NeZ} -\textipa{sfE} }
	{hour -\textsc{poss.pl} four -\textsc{poss.pl}}
	{its-hours {} its-four {}}
	{four hours(\textsc{possessive})}
	\item 
		\trigloss[preamble={\textipa{xEn@ NeZ@}  }]
	{\textipa{xEn} -\textipa{@} \textipa{NeZ} -\textipa{@} }
	{hour -\textsc{adpos.pl} four -\textsc{adp.pl}}
	{hours {} four {} }
	{four hours(\textsc{adpositive})}
\end{enumerate}

\subsubsection{Ordinals}
Ordinals do not agree with the nouns they modify in case and number.  Rather, they are formed by attaching to the beginning of nouns they modify as a clitic.

\textit{e.g.} 
\begin{enumerate}
	\item
{rising tide : \textipa{ziZIm} \\
second rising tide : \textipa{ak\h} + \textipa{ziZIm} $\rightarrow$ \textipa{aNziZIm} }

\item
	{fish: \textipa{S:eS:} \\
second fish: \textipa{ak\h} + \textipa{S:eS:} $\rightarrow$ \textipa{aNS:eS:} \\
third fish: \textipa{sa} + \textipa{S:eS:e} $\rightarrow$ \textipa{saS:eS:} \\
eighth fish: \textipa{esl} + \textipa{S:eS:} $\rightarrow$ \textipa{eslS:eS:} \\
25th fish: \textipa{HUtHeHU} + \textipa{S:eS:} $\rightarrow$ \textipa{HUtHeHUS:eS:}   }
\end{enumerate}

\section{Recursion}
\label{sec:recursion}
Nominal possession can be applied recursively, where the possession is applied first to the head, then to each possessor, applying the ordering rules for alienable/inalienable nouns.

\textit{e.g.}
\begin{enumerate}
	\item 
		\trigloss[preamble={\textipa{saIp\h{} NEwHIt\h{}}}]
		{\textipa{saip\h{}} -\N{} \textipa{NEwH} -\textipa{It\h{}} }
		{tail-fin {} color -\textsc{1.sg.poss} }
		{tail-fin {} its-color {} }
		{tail-fin's color}
	\item 
		\trigloss[preamble={\textipa{saImIt\h{} NEwHIt\h{}} S:eS:}]
		{\textipa{saim} -\textipa{It\h{}} \textipa{NEwH} -\textipa{It\h{}} \textipa{S:eS:} }
		{tail-fin -\textsc{1.sg.poss} color -\textsc{1.sg.poss} fish}
		{its-tail-fin {} its-color {} fish}
		{fish's tail-fin's color}
	\item 
		\trigloss[preamble={\textipa{saImIt\h{} NEwHIt\h{}} S:eS:It\h{} AmA}]
		{\textipa{saim} -\textipa{It\h{}} \textipa{NEwH} -\textipa{It\h{}} \textipa{S:eS:} -\textipa{It\h{}} \textipa{AmA}}
		{tail-fin -\textsc{1.sg.poss} color -\textsc{1.sg.poss} fish \textsc{1.sg.poss} parent}
		{its-tail-fin {} its-color {} its-fish {} parent}
		{parent's fish's tail-fin's color}
\end{enumerate}

We know in example 2 that the translation cannot be \textit{fish's color's tail-fin} because color is inalienable, so its possessor comes before it.  We can apply this same logic to example 3, also knowing that fish is alienable to parent, so it must be \textit{paren'ts fish's...} rather than \textit{fish's parent...}.

\section{Quantifiers and Definiteness}
Quantifiers and definite articles modify nouns and thus agree with the nouns they modify in number and case.  For the quantifier \textipa{nEm}, even if all of the noun in question is just one, they take the plural. Quantifiers and definite articles are listed after adjectives in use.
\\ \\
\textit{e.g. all}
\begin{enumerate}
	\item 
		\trigloss[preamble={\textipa{S:eS: SlIms nEm}}]
		{\textipa{S:eS:} -\N{} \textipa{SlIms} -\N{} \textipa{nEm} -\N{} }
		{fish {} red {} all {}}
		{fish {} red {} all {}}
            	{all red fish}
\end{enumerate}
\textit{e.g. definiteness}
\begin{enumerate}
	\item
\trigloss[preamble={\textipa{S:eS: SlIms sIm}}]
		{\textipa{S:eS:} -\N{} \textipa{SlIms} -\N{} \textipa{sIm} -\N{} }
		{fish {} red {} specific {}}
		{fish {} red {} specific {}}
            	{specific red fish}
	\item
\trigloss[preamble={\textipa{S:eS:A SlImsA sImAmat\h{}}}]
		{\textipa{S:eS:} -\textipa{A} \textipa{SlIms} -\textipa{A} \textipa{sIm} -\textipa{A} -\textipa{mat\h{}} }
		{fish -\textsc{sg.adp} red -\textsc{sg.adp} specific -\textsc{sg.adp} -textipa{io.post}}
		{fish {} red {} specific {} to}
            	{to the specific red fish}
\end{enumerate}


\section{Putting Things Together}
Here are some examples of how all of these rules work together.

\textit{pronouns + possessives}
\begin{enumerate}
	\item
		\trigloss[preamble={\textipa{S:eS:It\h{} wlAt\h{}}}]
		{\textipa{S:eS:} -\textipa{It\h{}} \textipa{wlat\h{}} }
		{fish -\textsc{sg.poss} me }
		{its-fish me}
            	{my fish}
	\item
		\trigloss[preamble={\textipa{S:eS:It\h{} Za}}]
		{\textipa{S:eS:} -\textipa{It\h{}} \textipa{Za} }
		{fish -\textsc{sg.poss} it }
		{its-fish it}
            	{its fish}
	\item
		\trigloss[preamble={\textipa{S:eS:It\h{} slAp\h{}}}]
		{\textipa{S:eS:} -\textipa{It\h{}} \textipa{slAp\h{}} }
		{fish -\textsc{sg.poss} \textsc{1.du.incl} }
		{its-fish you-and-me}
            	{our fish}
	\item
		\trigloss[preamble={\textipa{S:eS:It\h{} slAn!}}]
		{\textipa{S:eS:} -\textipa{It\h{}} \textipa{slAn!}} 
		{fish -\textsc{sg.poss} \textsc{1.du.excl} }
		{its-fish two-of-us-but-not-you}
            	{our fish}

\end{enumerate}

\textit{possessive + recursion + pronoun + adjective}
\begin{enumerate}
	\item 
		\trigloss[preamble={\textipa{S:eS:it\h{} saImsfE SlIms:fE slIxsfE wlAt\h{}}}]
		{\textipa{S:eS:} -\textipa{it\h} \textipa{saIm} -\textipa{sfE} \textipa{SlIms} -\textipa{sfE} \textipa{slIx} -\textipa{sfE} \textipa{wlAt\h} }
		{fish -\textsc{sg.poss} tail-fin -\textsc{pl.poss} red -\textsc{pl.poss} some -\textsc{pl.poss} \textsc{1.sg.nom} }
		{its-fish {} its-tail-fin -s red {} some {} me}
		{my fish's unspecified number of red tail-fins}
\end{enumerate}

\textit{recursion + case ending}
\begin{enumerate}
	\item
		\trigloss[preamble={\textipa{saImIt\h{} NEwHIt\h{} S:eS:It\h{} AmA:\R{}}}]
		{\textipa{saIm} -\textipa{It\h{}} \textipa{NEwH} -\textipa{It\h} \textipa{S:eS:} -\textipa{It\h{}} \textipa{AmA} -\textipa{A\R{}} }
		{tail-fin -\textsc{sg.poss} color -\textsc{sg.poss} fish \textsc{sg.poss} parent -\textsc{sg.acc} }
		{its-tail-fin {} its-color {} its-fish {} parent {}}
		{parent's fish's tail-fin's color}
\end{enumerate}

\textit{recursion + adjectives/plurals/numerals}
\begin{enumerate}
	\item 
		\trigloss[preamble={\textipa{saImIt\h{} NEwHIt\h{} S:eS: SlIms}}]
		{\textipa{saIm} -\textipa{It\h{}} \textipa{NEwH} -\textipa{It\h} \textipa{S:eS:} -\N{} \textipa{SlIms} -\N{} }
		{tail-fin -\textsc{sg.poss} color -\textsc{sg.poss} fish -\textsc{sg.nom} red -\textsc{sg.nom} }
		{its-tail-fin {} its-color {} fish {} red {}}
		{red fish's tail-fin's color}
	\item 
		\trigloss[preamble={\textipa{saImIt\h{} NEwHIt\h{} S:eS:It\h{} SlImsIt\h{} AmA}}]
		{\textipa{saIm} -\textipa{It\h{}} \textipa{NEwH} -\textipa{It\h} \textipa{S:eS:} -\textipa{It\h{}} \textipa{SlIms} -\textipa{It\h{}} \textipa{AmA} -\N{}}
		{tail-fin -\textsc{sg.poss} color -\textsc{sg.poss} fish -textsc{sg.poss} red -textsc{sg.poss} parent {} }
		{its-tail-fin {} its-color {} its-fish {} its-red {} parent {}}
		{parent's red fish's tail-fin's color}
	\item
		\trigloss[preamble={\textipa{saImwE NEwHwE S:eS:It\h{} AmA:\R{}}}]
		{\textipa{saIm} -\textipa{wE} \textipa{NEwH} -\textipa{wE} \textipa{S:eS:} -\textipa{It\h{}} \textipa{AmA} -\textipa{A\R{}} }
		{tail-fin -\textsc{du.poss} color -\textsc{du.poss} fish \textsc{sg.poss} parent -\textsc{sg.acc} }
		{its-tail-fin two its-color -s its-fish {} parent {}}
		{parent's fish's two tail-fins' colors}

\end{enumerate}

\textit{ordinal + adjective}
\begin{enumerate}
	\item 
		\trigloss[preamble={\textipa{aNS:eS: SlIms}}]
		{\textipa{aN}- \textipa{S:eS:} -\N{} \textipa{SlIms} }
		{second fish -\textsc{nom.sg} red }
		{second fish {} red}
		{second red fish}

\end{enumerate}


\textit{possessive + definiteness}
\begin{enumerate}
	\item 
		\trigloss[preamble={\textipa{S:eS:It\h{} sImIt\h{} wlAt\h{} }}]
		{\textipa{S:eS:} -\textipa{It\h} \textipa{sIm} -\textipa{It\h} \textipa{wlAt\h} }
		{fish -\textsc{sg.poss} specific -\textsc{sg.poss} me }
		{its-fish {} its-specific {} me}
		{my specific fish}

\end{enumerate}

\textit{possessive + definiteness + ordinal + adjective}

\begin{enumerate}
	\item 
		\trigloss[preamble={\textipa{aNS:eS:It\h{} SlImsIt\h{} sImIt\h{} wlAt\h{} }}]
		{\textipa{aN}- \textipa{S:eS:} -\textipa{It\h} \textipa{SlIms} -\textipa{It\H{}} \textipa{sIm} -\textipa{It\h} \textipa{wlAt\h} }
		{second fish -\textsc{sg.poss} red -\textsc{sg.poss} specific -\textsc{sg.poss} me }
		{second its-fish {} its-red {} its-specific {} me}
		{my specific second red fish}

\end{enumerate}

\textit{possessive + quantifier}
\begin{enumerate}
	\item 
		\trigloss[preamble={\textipa{S:eS: saImsfE nEmsfE }}]
		{\textipa{S:eS:} -\N{} \textipa{saIm} -\textipa{sfE} \textipa{nEm} -\textipa{sfE}}
		{fish -\textsc{nom.sg} tail-fin -\textsc{pl.poss} all -\textsc{pl.poss} }
		{fish {} its-tail-fin -s its-all {}}
		{all of the fish's tail-fins}

\end{enumerate}


\textit{possessive + quantifier + definiteness + ordinals + adjectives + numbers + recursion}
\begin{enumerate}
	\item 
		\trigloss[preamble={\textipa{S:eS:It\h{} nEmIt\h{} aNsaImwE SlImswE aNwE sImwE AmAm sIm:}}]
		{\textipa{S:eS:} -\textipa{It\h} \textipa{nEm} -\textipa{It\h} \textipa{aN}- \textipa{saIm} -\textipa{wE} \textipa{SlIms} -\textipa{wE} \textipa{aN} -\textipa{wE} \textipa{sIm} -\textipa{wE} \textipa{AmA} -\textipa{m} \textipa{sIm} -\textipa{m}}
		{fish -\textsc{sg.poss} all -\textsc{sg.poss} second- tail-fin -\textsc{du.poss} red -\textsc{du.poss} two -\textsc{du.poss} specific -\textsc{du.poss} parent -\textsc{du.nom} specific -\textsc{du.nom}}
		{its-fish {} its-all {} second- its-tail-fin {} its-red {} its-two {} its-specific {} parent -s specific {}}
		{two specific red second tail-fins possessed by all the fishes belonging to two specific parents}

\end{enumerate}



\section{Phonological Rules}
Stops are only allowed at the ends of words in this language, so if through any suffixation a stop finds its way into the middle or beginning of a word, it changes to a voiced nasal (while place of articulation remains the same).

\textit{e.g.} \\
hour (\textsc{nom.sg}) : \textipa{xEt\h{}} \\
some hours (\textsc{nom.pl}): \textipa{xEt\h{}}+\textipa{He} \textipa{slIxHe} $\rightarrow$ \textipa{xEnHe slIxHe} \\
hour (\textsc{acc.sg}): \textipa{xEt\h{}}+\textipa{\R{}} $\rightarrow$ \textipa{xEn\R{}} \\


\end{document}

 \item 
	    \trigloss[preamble={\textipa{saZIm eINa}}]
            	{\textipa{saZIm} -$\varnothing$ \textipa{eIN} -\textipa{a} }
		{{falling tide} -\textsc{nom.sg} begin -\textsc{3.sg.prs} }
		{{falling tide} {} begin -s }
            	{It is just past 11 am}




