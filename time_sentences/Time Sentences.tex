\documentclass{article}
\usepackage[utf8]{inputenc}
\usepackage[noglossbreaks]{covington}
\usepackage{tipa}
\usepackage{amssymb}
\usepackage{geometry}
 \geometry{
 a4paper,
 total={170mm,257mm},
 left=20mm,
 top=20mm,
 }

\title{Time Sentences}
\author{Alison Hau\vspace{-2ex}}
\date{}

\begin{document}
\maketitle
 
\section{Numbers}
The number system of this language (name undetermined) is in base 12, because each hand has 6 fingers. Numbers 0-12 have their own words, numbers 13-15 are compounds with some irregularity, and all numbers beyond are also compounds formed the same way but regularly.  Compounds take the form \textit{number + } \textit{He} \textbf{(\textit{and} used for adding numbers) + number}.  24 is its own word, and other multiples of 12 are compounds of the form \textit{k twelve}. \\

Table \ref{table:1} has some examples of numbers.  \\


\begin{table}[h]
\centering
\begin{tabular}{l l l }

\hline
    & Language (IPA) & English (literal) \\
\cline{2-3}
1       &  \textipa{HU}   & one      \\
2       & \textipa{ak$^h$}    & two       \\
3       & \textipa{sa}    & three      \\
4       & \textipa{NeZ}     & four      \\
5       & \textipa{mUt$^h$}      & five       \\
6       & \textipa{\OE mp$^h$}   & six      \\
7       & \textipa{ei@}        & seven       \\
8       & \textipa{esl}     & eight      \\
9       & \textipa{Hi}     & nine      \\
10      & \textipa{iR$^h$}      & ten       \\
11      & \textipa{fUf}   & eleven      \\
12       & \textipa{at$^h$}        & twelve       \\
13       & \textipa{anHeU}      & twelve and one      \\
14       & \textipa{anHea}      & twelve and two      \\
15       & \textipa{anHesa}       & twelve and three       \\
16       & \textipa{anHeNeZ}     & twelve and four      \\
17       & \textipa{anHemUt$^h$}        & twelve and five       \\
18       & \textipa{anHe\OE mp$^h$}      & twelve and six      \\
19       & \textipa{anHe:i@}      & twelve and seven      \\
20       & \textipa{anHe:sl}       & twelve and eight       \\
21       & \textipa{anHeHi}     & twelve and nine      \\
22       & \textipa{anHeiR$^h$}        & twelve and ten       \\
23       & \textipa{anHefUf}      & twelve and eleven       \\
	24       & \textipa{hU\t{tS}}     & twenty four      \\
36       & \textipa{sa:t$^h$}            & three twelve \\
37       & \textipa{sa:nHeHU}             & three twelve and one \\
\hline
\end{tabular}
\caption{Some number examples}
\label{table:1}
\end{table}

\section{Time}
% syntax of number/time phrases
% syntax of carrier sentences
\indent The numbers six and 12 are important to this society, as each person has 12 fingers, or six sets of webbed pairs (two thumbs, index and middle, and ring and pinky fingers, to use human analogues).  Each year is divided into five months, each named after an animal historically hunted in that month.  Each month has six \textit{\textipa{aIHE}} ("weeks") of twelve \textit{\textipa{a:NHE}} (days) each and with 5 days at the end of the year not in any month.  The days of the week are named after fingers on the hand, starting from right to left. \\

The time system is based on the tides, dating back to when this civilization lived close enough to the coast to be affected by the tides.  Even though they have moved away from the coasts toward open ocean and the deeps, they still use the established system of keeping time. \\ 

Each day is divided into 4 sections of six hours, originally based off the tidal schedule (see Table \ref{table:2}).  Each hour has (about) 60 minutes (like human hours).  The first section of the day starts around what we'd call 5:00 AM.  Time is told by how many hours into a section have passed.\\

\begin{table}[h]
\centering
\begin{tabular}{l l l }

\hline
 Time (human)  & Language (IPA) & English (literal) \\
\hline
5:00-11:00       &  \textipa{ziZIm}   & rising tide     \\
11:00-16:00       &  \textipa{saZIm}        & falling tide      \\
16:00-23:00       &  \textipa{aNziZIm}    & second rising tide   \\
23:00-5:00       &  \textipa{aNsaZIm}     & second falling tide      \\
\hline
\end{tabular}
\caption{The four sections of the day}
\label{table:2}
\end{table}

\subsection{Sentences}
Time sentences take the form of a "has" sentence.  A time sentence starts with which tide it is in the nominative, followed by the number of hours (or hour, singular) into the tide in the accusative, then the singular third-person form of the verb \textit{to have}.  The hours can be modified by \textit{and half} to indicate that it is near the middle of the hour, as it's hard to keep track of exact minutes underwater.  It is also valid to use the present third-person verb \textit{begins} or \textit{ends} rather than the hours + \textit{to have} if a new tide has just begun or will begin soon respectively.  The \textit{to have} or \textit{one} (when only one hour has passed since a new tide) are sometimes elided.

\begin{enumerate}
	\item 
	    \trigloss[preamble={\textipa{saZIm eINa}}]
            	{\textipa{saZIm} -$\varnothing$ \textipa{eIN} -\textipa{a} }
		{{falling tide} -\textsc{nom.sg} begin -\textsc{3.sg.prs} }
		{{falling tide} {} begin -s }
            	{It is just past 11 am}
            
    \item 
	    \trigloss[preamble={\textipa{ziZIm sWla}}]
            	{\textipa{ziZIm} -$\varnothing$ \textipa{sWl} -\textipa{a} }
		{{rising} -\textsc{nom.sg} end -\textsc{3.sg.prs} }
		{{rising tide} {} end -s }
		{It is almost 11 am}
            
    \item 
	    \trigloss[preamble={\textipa{aNsaZIm xEn\*r  NeZ\*r  Insa}}]
            {\textipa{aN}- \textipa{saZIm} -$\varnothing$ \textipa{xEn} -\textipa{\*r} \textipa{NeZ} -\textipa{\*r} \textipa{Ins} -\textipa{a} }
		{two- {falling tide} -\textsc{nom.sg} hour -\textsc{acc.pl} four -\textsc{acc.pl}  have -\textsc{3.sg.prs} }
		{second- {falling tide} {} hour -s four {} has {} }
		{It is 3 am}

    \item 
	    \trigloss[preamble={\textipa{aNziZIm xEn\*r  aNHemUn\*r Insa}}]
	    {\textipa{aN}- \textipa{ziZIm} -$\varnothing$ \textipa{xEn} -\textipa{\*r} \textipa{aN}  -\textipa{He} -\textipa{mUn} -\textipa{\*r} \textipa{Ins} -\textipa{a} }
		{two- {falling tide} -\textsc{nom.sg} hour -\textsc{acc.pl} two -and -half -\textsc{acc.pl}  have -\textsc{3.sg.prs} }
		{second- {rising tide} {} hour -s two and half {} has {} }
		{It is 7:30 pm}

    \item 
	    \trigloss[preamble={\textipa{aNziZIm xEn\*r  aNHemUn\*r}}]
		{\textipa{aN}- \textipa{ziZIm} -$\varnothing$ \textipa{xEn} -\textipa{\*r} \textipa{aN}  -\textipa{He} -\textipa{mUn} -\textipa{\*r} }
		{two- {falling tide} -\textsc{nom.sg} hour -\textsc{acc.pl} two -and -half -\textsc{acc.pl} }
		{second- {rising tide} {} hour -s two and half {} }
		{It is 7:30 pm}
            

    \item 
	    \trigloss[preamble={\textipa{saZIm xEnA\*r HUar Insa}}]
		{\textipa{saZIm} -$\varnothing$ \textipa{xEn} -\textipa{A\*r} \textipa{HU} -\textipa{A\*r} \textipa{Ins} -\textipa{a}}
		{{falling tide} -\textsc{nom.sg} hour -\textsc{acc.sg} one -\textsc{acc.sg} have -\textsc{3.sg.prs}}
		{{falling tide} {} hour {} one {} has {} }
		{It is noon}
		
    \item 
	    \trigloss[preamble={\textipa{saZIm xEnA\*r Insa}}]
		{\textipa{saZIm} -$\varnothing$ \textipa{xEn} -\textipa{A\*r} \textipa{Ins} -\textipa{a}}
		{{falling tide} -\textsc{nom.sg} hour -\textsc{acc.sg} have -\textsc{3.sg.prs}}
		{{falling tide} {} hour {} has {}}
		{It is noon}
     \item 
	    \trigloss[preamble={\textipa{saZIm xEnA\*r }}]
		{\textipa{saZIm} -$\varnothing$ \textipa{xEn} -\textipa{A\*r} }
		{{falling tide} -\textsc{nom.sg} hour -\textsc{acc.sg} }
		{{falling tide} {} hour {} }
		{It is noon}
    \item 
	    \trigloss[preamble={\textipa{ziZIm xEni\*r  aNi\*r Insa}}]
		{\textipa{ziZIm} -$\varnothing$ \textipa{xEn} -\textipa{i\*r} \textipa{aN}  -\textipa{i\*r} \textipa{Ins} -\textipa{a}}
		{{falling tide} -\textsc{nom.sg} hour -\textsc{acc.du} two -\textsc{acc.du}  have -\textsc{3.sg.prs} }
		{{rising tide} {} hour -s two {} has {} }
		{It is 7:00 am}
\end{enumerate}

Comparison of examples 4 \& 5 and 6, 7, \& 8 demonstrate omission of \textit{to have} and \textit{one}.  This language does not mark definiteness or gender, but does mark case and number (singular, dual, plural).

\end{document}


